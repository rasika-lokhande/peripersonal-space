One of the fundamental human abilities involved in human-environment interaction is extending one's arm to make physical contact with an object. To perform a successful reaching action, the agent's cognitive system must encode the spatial location of the target. Peri-personal space is the region of space surrounding the agent's body where reach targets are located. One of the characteristics of peri-personal space is that the objects in this region of space are encoded with respect to the region of the body-surface that is in close proximity to the object. In addition, action-relevant body parameters, such as the position and structure of the limbs, are encoded in a body representation which is constructed by integrating visual and proprioceptive sensory information about the body. 

Considering the above, this thesis investigates the following two questions. Firstly, is the spatial information of the reach target encoded relative to the body part located proximally, even if that body part is not the effector performing the reach action? Secondly, if the answer to the first question is affirmative, how does the integration of visuo-proprioceptive signals regarding the proximal body part contribute to the spatial encoding of the target? To investigate these questions, two  experiments were conducted in an immersive virtual reality set-up, in which a reach task was performed while the action hand was rendered invisible. Visual and proprioceptive information of the proximal body part was manipulated as the experimental conditions, and the accuracy of reaches was measured. The results of the two experiments indicate that the spatial encoding of the reach target occurs relative to the proximal body-part, and that the integration of the proximal body-part's visual and proprioceptive sensory signals underlies the spatial encoding of the reach target's location. However, the results further suggest that the spatial encoding of the target occurs relative to the body-part location indicated by its visual signal, as opposed to the body-part position inferred after the integration of multi-sensory signals.









%Reaching one's arm out to make contact with an object is one of the fundamental human capabilities involved in human-environment interaction. To perform a successful reaching action, the spatial location of the target must be encoded in the cognitive system of the agent. Targets of reach action are located in the region of space surrounding the agent's body- termed as \textit{peri-personal space}. One of the characteristics of peri-personal space is that the objects located in this region of space are encoded with respect to the region of the body-surface which is in proximity to the target. Moreover, body parameters relevant for action, such as position and structure of the limbs, are encoded in a body representation which may be constructed from the integration of visual and proprioceptive sensory information regarding the body. 

%Considering the above, we investigated the following two questions in this thesis. Firstly, is the spatial information of the reach target encoded with respect to the body-part situated in its proximity, even if that body-part is not the effector undertaking the reach action? Secondly, if yes, how is the integration of visuo-proprioceptive signals regarding the proximal body-part involved in the spatial encoding of the target? To investigate these questions, two experiments were conducted in an immersive virtual reality environment, in which a reach task was implemented while the action hand was rendered invisible. Visual and proprioceptive information of the proximal body-part were also manipulated as the experimental conditions, and the accuracy of the reaches was measured. The results of the two experiments suggest that the spatial encoding of the reach target occurs with respect to the proximal body-part, and that integration of proximal body-part's visual and proprioceptive sensory signals underlies the encoding of the reach target's spatial location. However, the results further suggest that the spatial encoding of the target occurs with respect to the body-part location indicated by its visual signal as opposed to the body-part location estimated as a result of multi-sensory integration process.
